%%%%%%%%%%%%%%%%%%%%%%%%%%%%%%%%%%%%%%
% KnitR Poster Template
% Daniel Guest 
% Updated September 2017
% Adapted from template from Nathaniel Johnston for use with KnitR...
%	- http://www.nathanieljohnston.com/2009/08/latex-poster-template/
%%%%%%%%%%%%%%%%%%%%%%%%%%%%%%%%%%%%%%

\documentclass[final]{beamer}\usepackage[]{graphicx}\usepackage[]{color}
%% maxwidth is the original width if it is less than linewidth
%% otherwise use linewidth (to make sure the graphics do not exceed the margin)
\makeatletter
\def\maxwidth{ %
  \ifdim\Gin@nat@width>\linewidth
    \linewidth
  \else
    \Gin@nat@width
  \fi
}
\makeatother

\definecolor{fgcolor}{rgb}{0.345, 0.345, 0.345}
\newcommand{\hlnum}[1]{\textcolor[rgb]{0.686,0.059,0.569}{#1}}%
\newcommand{\hlstr}[1]{\textcolor[rgb]{0.192,0.494,0.8}{#1}}%
\newcommand{\hlcom}[1]{\textcolor[rgb]{0.678,0.584,0.686}{\textit{#1}}}%
\newcommand{\hlopt}[1]{\textcolor[rgb]{0,0,0}{#1}}%
\newcommand{\hlstd}[1]{\textcolor[rgb]{0.345,0.345,0.345}{#1}}%
\newcommand{\hlkwa}[1]{\textcolor[rgb]{0.161,0.373,0.58}{\textbf{#1}}}%
\newcommand{\hlkwb}[1]{\textcolor[rgb]{0.69,0.353,0.396}{#1}}%
\newcommand{\hlkwc}[1]{\textcolor[rgb]{0.333,0.667,0.333}{#1}}%
\newcommand{\hlkwd}[1]{\textcolor[rgb]{0.737,0.353,0.396}{\textbf{#1}}}%
\let\hlipl\hlkwb

\usepackage{framed}
\makeatletter
\newenvironment{kframe}{%
 \def\at@end@of@kframe{}%
 \ifinner\ifhmode%
  \def\at@end@of@kframe{\end{minipage}}%
  \begin{minipage}{\columnwidth}%
 \fi\fi%
 \def\FrameCommand##1{\hskip\@totalleftmargin \hskip-\fboxsep
 \colorbox{shadecolor}{##1}\hskip-\fboxsep
     % There is no \\@totalrightmargin, so:
     \hskip-\linewidth \hskip-\@totalleftmargin \hskip\columnwidth}%
 \MakeFramed {\advance\hsize-\width
   \@totalleftmargin\z@ \linewidth\hsize
   \@setminipage}}%
 {\par\unskip\endMakeFramed%
 \at@end@of@kframe}
\makeatother

\definecolor{shadecolor}{rgb}{.97, .97, .97}
\definecolor{messagecolor}{rgb}{0, 0, 0}
\definecolor{warningcolor}{rgb}{1, 0, 1}
\definecolor{errorcolor}{rgb}{1, 0, 0}
\newenvironment{knitrout}{}{} % an empty environment to be redefined in TeX

\let\hlesc\hlstd \let\hlpps\hlstd \let\hllin\hlstd \let\hlslc\hlcom \let\hlppc\hlcom
\usepackage{alltt}
\usepackage[scale=1.20]{beamerposter}
\usepackage{graphicx} % Importing images  
\usepackage{tabularx} % Better tables
\usepackage{float} % Floats, obviously
\usepackage{booktabs} % Better tables
\usepackage{amsmath} % Better tables
\usepackage{hyperref} % URLs
\usepackage[neverdecrease]{paralist} % Control over list indentation
\usepackage[english]{babel}
\usepackage{blindtext}

%-----------------------------------------------------------
% Define the column width and poster size
% To set effective sepwid, onecolwid and twocolwid values, first choose how many columns you want and how much separation you want between columns
% The separation I chose is 0.024 and I want 4 columns
% Then set onecolwid to be (1-(4+1)*0.024)/4 = 0.22
% Set twocolwid to be 2*onecolwid + sepwid = 0.464
% Formulae for n columns:
%	onecolwid = (1/n)*(1-(n+1)*sepwid)
%   twocolwid = 2*onecolwid + sepwid
%   threecolwid = 3*onecolwid + 2*sepwid
%-----------------------------------------------------------

\newlength{\sepwid}
\newlength{\minisepwid}
\newlength{\onecolwid}
\newlength{\twocolwid}
\newlength{\threecolwid}
\setlength{\paperwidth}{48in} 
\setlength{\paperheight}{36in}
\setlength{\sepwid}{0.024\paperwidth}
\setlength{\minisepwid}{0.06\onecolwid}
\setlength{\onecolwid}{0.22\paperwidth}
\setlength{\twocolwid}{0.464\paperwidth}
\setlength{\threecolwid}{0.708\paperwidth}
\setlength{\topmargin}{-0.5in}
\usetheme{confposter}
\usepackage{exscale}
\setdefaultleftmargin{1cm}{2cm}{}{}{}{}

%-----------------------------------------------------------
% The next part fixes a problem with figure numbering. Thanks Nishan!
% When including a figure in your poster, be sure that the commands are typed in the following order:
% \begin{figure}
% \includegraphics[...]{...}
% \caption{...}
% \end{figure}
% That is, put the \caption after the \includegraphics
%-----------------------------------------------------------

\usecaptiontemplate{
\small
\structure{\insertcaptionname~\insertcaptionnumber:}
\insertcaption}

%-----------------------------------------------------------
% Define colours (see beamerthemeconfposter.sty to change these colour definitions)
%-----------------------------------------------------------

\setbeamercolor{block title}{fg=UMNMaroon,bg=white}
\setbeamercolor{block body}{fg=black,bg=white}
\setbeamercolor{block alerted title}{fg=UMNGold!10,bg=UMNMaroon!70}
\setbeamercolor{block alerted body}{fg=black,bg=UMNGold!10}

%-----------------------------------------------------------
% Custom definitions
%-----------------------------------------------------------



%-----------------------------------------------------------
% Bibliography Stuff
%-----------------------------------------------------------

\usepackage[backend=biber]{biblatex}
\bibliography{/home/daniel/Desktop/logbook/Citations/bibliography} % Point to master bibliography

%-----------------------------------------------------------
% Name and authors
%-----------------------------------------------------------

\title{An example poster using KnitR and LaTeX}
\author{Daniel Guest}
\institute{University of Somewhere, Department of Something, One of the Labs}

%-----------------------------------------------------------
% Start the poster
%-----------------------------------------------------------
\IfFileExists{upquote.sty}{\usepackage{upquote}}{}
\begin{document}







\begin{frame}[t, fragile]
	\begin{columns}[T] % Overall <Begin>
		\begin{column}{\sepwid}\end{column} % Spacer column
		\begin{column}{\onecolwid} % Column 1 <Begin>
				\begin{block}{Introduction}
						\begin{itemize}
							\item This is an example poster created in \LaTeX{}
							\item A few \LaTeX{} packages and other pieces of code and software are used to make this all work:
							\begin{itemize}
								\item The beamer and beamerposter packages are integral to this poster
								\item A slightly modified version of the beamerposter theme available from \url{http://www.nathanieljohnston.com/2009/08/latex-poster-template/} was used to style the visual aspects of the poster
								\item KnitR was used to allow the output of R code (text, numbers, and figures) to be included directly in the poster
							\end{itemize}
						\end{itemize}
				\end{block}
				\begin{alertblock}{Motivations}
						Why use \LaTeX{} to make a poster? What advantages does it offer over PowerPoint or other alternatives?
						\begin{itemize}
								\item LaTeX{} emphasizes a clear separation of content from form, allowing you to focus on the ``what'' rather than the ``how'' 
								\item KnitR allows for syntax-highlighted R code, or the output of R code, to be included directly in the poster 
								\item LaTeX{} has great tools for high-quality typesetting of math, like $\int 2x^2 dx$
								\item Bibliography systems like biblatex make managing citations and including them in your poster easy
								\item If there's something unique you need to be able to do (like write International Phonetic Alphabet, or draw diagrams), you can likely extend \LaTeX{} through packages to do so
						\end{itemize}
				\end{alertblock}
				\begin{block}{How it works}
						\begin{itemize}
							\item The files
								\begin{itemize}
									\item \texttt{beamerthemconfposter.sty} --- contains definitions of the title and blocks, as well as colors and theme options
									\item \texttt{beamerposter.sty} --- provides the beamerposter package
									\item \texttt{poster.Rnw} --- the source file, which is turned into a .tex file by KnitR
								\end{itemize}
							\item The source file
								\begin{itemize}
							\item The poster is composed of a title and columns, with each column being subdivided into blocks
							\item Blocks contain the content of the poster, and come in two flavors -- normal (like the first block, ``Introduction'') and alert (like the this block, ``Motivations'')
							\item R code is delimited by special characters \texttt{<<>>} and \texttt{@}
								\end{itemize}
						\end{itemize}
				\end{block}
		\end{column} % Column 1 <End>
		\begin{column}{\sepwid}\end{column} % Spacer column
		\begin{column}{\twocolwid} % Column 2/3 <Begin>
				\begin{block}{Some examples}
				KnitR supports high-quality syntax-highlighting of R (and many other languages), as well as directly including the output of R in the output document
\begin{knitrout}\tiny
\definecolor{shadecolor}{rgb}{0.969, 0.969, 0.969}\color{fgcolor}\begin{kframe}
\begin{alltt}
\hlstd{my_data} \hlkwb{=} \hlstd{iris}
\hlstd{my_summary} \hlkwb{=} \hlstd{my_data} \hlopt \hlkwd{group_by}\hlstd{(Species)} \hlopt \hlkwd{summarize}\hlstd{(}\hlkwc{sepallen}\hlstd{=}\hlkwd{mean}\hlstd{(Sepal.Length),}
                                                         \hlkwc{sepalwid}\hlstd{=}\hlkwd{mean}\hlstd{(Sepal.Width),}
                                                         \hlkwc{petallen}\hlstd{=}\hlkwd{mean}\hlstd{(Petal.Length),}
                                                         \hlkwc{petalwid}\hlstd{=}\hlkwd{mean}\hlstd{(Petal.Width))}
\hlstd{my_summary}
\end{alltt}
\begin{verbatim}
## # A tibble: 3 x 5
##      Species sepallen sepalwid petallen petalwid
##       <fctr>    <dbl>    <dbl>    <dbl>    <dbl>
## 1     setosa      5.0      3.4      1.5     0.25
## 2 versicolor      5.9      2.8      4.3     1.33
## 3  virginica      6.6      3.0      5.6     2.03
\end{verbatim}
\end{kframe}
\end{knitrout}

This functionality includes graphs!
\begin{knitrout}\tiny
\definecolor{shadecolor}{rgb}{0.969, 0.969, 0.969}\color{fgcolor}\begin{kframe}
\begin{alltt}
\hlkwd{ggplot}\hlstd{(my_data,} \hlkwd{aes}\hlstd{(}\hlkwc{x}\hlstd{=Sepal.Length,} \hlkwc{y}\hlstd{=Sepal.Width,} \hlkwc{color}\hlstd{=Petal.Length))} \hlopt{+}
     \hlkwd{geom_point}\hlstd{()} \hlopt{+} \hlkwd{facet_wrap}\hlstd{(}\hlopt{~}\hlstd{Species)}
\end{alltt}
\end{kframe}
\includegraphics[width=\linewidth]{figure/iris_example_2-1} 

\end{knitrout}

And, there's a variety of options to control whether or not your R code is visible or hidden and the size and characteristics of your output text and/or graphs. For example, here's a graph using the airquality data set in R, but the actual R code that generated the graph hidden in the output document.

\begin{knitrout}
\definecolor{shadecolor}{rgb}{0.969, 0.969, 0.969}\color{fgcolor}\begin{kframe}


{\ttfamily\noindent\itshape\color{messagecolor}{\#\# Picking joint bandwidth of 2.65}}\end{kframe}
\includegraphics[width=\linewidth]{figure/airquality_example_1-1} 

\end{knitrout}
				\end{block}
		\end{column} % Column 2/3 <End>
		\begin{column}{\sepwid} \end{column} % Spacer column
		\begin{column}{\onecolwid} % Column 4 <Begin>
				\begin{block}{Resources}
					\begin{itemize}
						\item \url{http://www.personal.kent.edu/~rmuhamma/Systems/latex.html} --- Big collection of \LaTeX{} resources
						\item \url{https://yihui.name/knitr/} --- overview of KnitR
						\item \url{https://yihui.name/knitr/options/} --- KnitR options to modify the behavior of R code chunks
					\end{itemize}
				\end{block}
				\begin{block}{An example compilation in Linux}
\begin{knitrout}\small
\definecolor{shadecolor}{rgb}{0.969, 0.969, 0.969}\color{fgcolor}\begin{kframe}
\noindent
\ttfamily
\hlstd{Rscript\ }\hlopt{{-}}\hlstd{e\ }\hlstr{"library(knitr);\ knitr('./poster.Rnw')"}\hlstd{\hspace*{\fill}\\
latexmk\ }\hlopt{{-}}\hlstd{pdf\ poster.tex}\hspace*{\fill}
\mbox{}
\normalfont
\end{kframe}
\end{knitrout}
				\end{block}
				\begin{block}{More examples of nice things \LaTeX{} can do}
					\begin{itemize}
							\item Math typesetting
									\begin{equation}
											\nabla \mathbf{f} = \frac{\partial\mathbf{f}}{\partial x}\mathbf{\hat{i}} + \frac{\partial\mathbf{f}}{\partial y}\mathbf{\hat{j}} + \frac{\partial\mathbf{f}}{\partial z}\mathbf{\hat{k}}
									\end{equation}
									\begin{equation}
											f(\zeta) = \int_{-\infty}^{\infty}f(x)e^{-2\pi i x \zeta} dx
									\end{equation}
							\item Tables
									\begin{table}
									\small
									\centering
									\begin{tabular}{llll}
									\toprule
									\textbf{Parameter} & \textbf{Texas} & \textbf{Minnesota} \\
									\midrule
									Population (mil.) & 27 & 6 \\
									Median income (thou. \$) & 56 & 68\\
									\bottomrule
									\end{tabular}
									\end{table}
					\end{itemize}
				\end{block}
		\end{column} % Column 4 <End>
		\begin{column}{\sepwid}\end{column} % Spacer column
	\end{columns} % Overall <End>
\end{frame}
\end{document}
